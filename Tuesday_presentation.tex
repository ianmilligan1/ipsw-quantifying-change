% Modelo de slides para projetos de disciplinas do Abel
\documentclass[10pt]{beamer}

\usetheme[progressbar=frametitle]{metropolis}
\usepackage{appendixnumberbeamer}
\usepackage[numbers,sort&compress]{natbib}
\bibliographystyle{plainnat}

\usepackage{booktabs}
\usepackage[scale=2]{ccicons}

\usepackage{xspace}
\newcommand{\themename}{\textbf{\textsc{metropolis}}\xspace}

\title{IPSW - Modelling Change of Website Archives}
\subtitle{Group 4}
\date{\today}
%\date{}
\author{Authors}
%\institute{UFPR - Disciplina - Semestre}
% \titlegraphic{\hfill\includegraphics[height=1.5cm]{logo.pdf}}

\begin{document}

\maketitle

\begin{frame}{Table of contents}
  \setbeamertemplate{section in toc}[sections numbered]
  \tableofcontents[hideallsubsections]
\end{frame}

\section{Introduction}

\begin{frame}[fragile]{The goal}

\begin{itemize}
	\item Construct and compare different metrics to quantify domain changes over time,
	\item Determine a single quantitative measure to describe magnitude of the change in the domain since the previous time-step.
	\end{itemize} 
\begin{align}
	\begin{split}
		\sigma(t) &= (\text{change in links})w_1 + (\text{change in text})w_2 \\
		&\qquad\qquad+ (\text{change in content management server})w_3,
	\end{split}
\end{align}
where $t$ is time, and $w_1$, $w_2$, and $w_3$ weight the relevant contributions of URL changes, text changes, and CMS changes.
\end{frame}

\begin{frame}{Current methods}
 
\begin{itemize}
	\item Meaningful changes determined by comparing thumbnails manually. 
	\vspace{3ex}
	\item Could automate this by using image analysis to quantify the difference between website thumbnails at two time points. 
\end{itemize}
\end{frame}

\begin{frame}{Game plan}
\begin{itemize}
\item Run code to compare text.
\item Do image analysis on thumbnails.
\item Take link data and compare lists at different times:
\begin{itemize}
\item Internal vs. external links.
\item Obtain $a$, $b$, and $c$. 
\item What is the best timestep?
\end{itemize}
\item Determine whether the content management server (CMS) has changed.
\item Look at different weightings - how best to choose these? We don't want to double-count changes.
\item Run test cases.
\item Look at the variability in change over time. What is the distribution?
\item Compare measures for looking at the difference between URLS and text.
\end{itemize}

\end{frame}

\begin{frame}

\begin{itemize}
\item Trying to quantify change using text, thumbnails and links.
\item Lots of metrics about the how the text differs and some of these are similar.
\begin{itemize}
\item There is one that is overly sensitive but there is still one timestamp that says there is absolutely no change and so could still be useful.
\end{itemize}  
\item Thumbnails obtained using the wayback archive which renders the homepage and takes a screenshot.
\begin{itemize}
\item We've used a metric that looks at structural similarity instead of just pixel to pixel which is good. 
\item We've had a problem with the website not always rendering and giving us just a white page which obviously causes a huge change. This needs to be accounted for tomorrow.
\end{itemize}
\end{itemize}
\end{frame}

\begin{frame}
\begin{itemize}
\item The link data has been analysed
\item \begin{itemize}
\item Unfortunately the dates for these data is shorter than the text data frame so it is difficult to get a good comparison. 
\item We have Ian, the history professor on this task.
\end{itemize}
\item Graphs of links
\begin{itemize}
\item One last thing we were thinking of doing is getting the internal links within a whole domain instead of just the homepages, seeing how the structure of the graph of links between them changes.
\item This is more of a structural change than a content change, which could be useful for rapidly updating websites such as news websites and blogs whose words change rapidly but fairly meaninglessly.
\end{itemize}
\end{itemize}

\end{frame}

\end{document}