\documentclass[10pt, a4paper]{article}
\usepackage{amsmath}
\usepackage{amsfonts}
\usepackage[english]{babel}
\usepackage[utf8]{inputenc}
\usepackage{color}
\usepackage{graphicx}
\usepackage[version=4]{mhchem}
\usepackage{rotating}
\renewcommand{\baselinestretch}{1}
\usepackage{fullpage}
\usepackage{mathtools}
\usepackage{caption}
\usepackage{tikz}
\usepackage{standalone}
\usetikzlibrary{decorations.pathreplacing}
\usetikzlibrary{arrows}
\usetikzlibrary{fadings}
\usepackage{parskip}
\usepackage{import}
\setlength{\parindent}{0pt}


\begin{document}

\title{IPSW - Modelling Change of Website Archives}
\author{Group 4}
\date{\vspace{-5ex}}
\maketitle

\section{Introduction}

\subsection{Goal}
Our goal is to investigate a novel metric for domain change that focuses on three branches of indicators, namely, changes in webpage text, changes in the links extending from the webpage and change to the homepage thumbnails. We propose that the  magnitude  of the total change in the domain from time-step $n-1$ to time-step $n$ can be modelled as 
\begin{equation}
\sigma(n) = \left(1-\frac{\Delta\, \,\text{links}}{\Sigma\,\,\text{links}}\right)w_1 + (\text{change in text})w_2 + (\text{change in content management server})w_3,
\end{equation}
where $w_1$, $w_2$, and $w_3$ weight the relevant contributions of URL changes, text changes, and CMS changes, respectively.

\section{Obtaining the data}
\subsection{Political party data}
\subsection{Pan Am Games 2015}

\section{Text}
We obtain the text from the homepage at every point in time. Our goal is to compare the text from one time-point to the next and quantify how much the text has changed between these measurements. There are a variety of metrics within the literature, in particular we explore the metrics described by Kwon et~al.  \cite{kwon2006precise}.
\subsection{Metrics}
\textit{\textbf{Byte-wise comparison metric}}:
Compares two webpages sequentially character by character. The metric then returns 1 if any change has occurred and zero otherwise \cite{brewington2000dynamic, cho1999evolution,kim2005empirical}. This means that it returns 1 for even the most trivial of changes, for example, adding a blank space. This metric is thereby over-sensitive and does not provide particularly meaningful insight into the change between two strings of text. However, the byte-wise metric is useful in limiting the pages of interest, namely, if at any time-step the metric is 0, we know that there have been absolutely no changes at all and hence we need not explore these events further.

\textit{\textbf{TF.IDF cosine distance}}:
TF.IDF is shorthand for term frequency-inverse document frequency and is used to quantify how important a word is to a document of text. The underlying concept is that relevant words are not necessarily the most frequent words, for example, if considering book reviews, the words ``character'' or ``plot'' might appear very frequently, but do not give valuable insight to summarise the review. 

This metric is calculated by finding the TF (term frequency) of a word, namely the frequency of a word in a document. We also find the IDF (inverse document frequency) of a word, which is the measure of how significant that term is in the collection of documents. By combining these concepts, we obtain TF.IDF weighted vectors to represent the content of each document and the metric value is calculated as the cosine distance between them \cite{salton1986introduction}. 

TF.IDF has evident success in search engine algorithms to shift the definition of word-value from frequency to relevance \cite{beel2016paper}.

%\begin{equation}
%D_{\cos} = 1-\frac{\boldsymbol{p}\cdot \boldsymbol{p}'}{||\boldsymbol{p}||_2||\boldsymbol{p}'||_2}.
%\end{equation}
\textit{\textbf{Word distance}}:
The word distance metric calculates how many of the words in a document have changed \cite{ntoulas2004s}. This is done by counting the number of common words in each document and normalising with respect to the total number of words in the two documents. 
%\begin{equation}

Although less sensitive than byte-wise, both TF.IDF and the word distance metric are unable to account for change in word order. 
%	D_{WD} = 1- \frac{2\cdot\vert\text{common words}\vert}{m+n},
%\end{equation}

\textit{\textbf{Levenshtein distance}}:
The Levenshtein distance between two strings is the minimum number of single character edits, (in other words substitutions, insertions or deletions) required to transform one string into the other \cite{levenshtein1996}. For example, the Levenshtein distance between ``test'' and ``tent'' is 1, due to the single substitution of ``s'' to ``n''.

We calculated the Levenshtein distance between the text of two webpages and normalised this value according to the maximum Levenshtein distance, namely the length of the longer string. 

\section{Thumbnails}

In addition to changes in website text and structure, meaningful change in domain is often reflected by a change in the visual structure of the page. We can generate a website snapshot at a recorded point in time using the Wayback Machine (\texttt{www.archive.org/web}) which is a digital archive of the web. Hence, a promising approach to quantify the change in a web domain, is found in applying image analysis techniques to detect the similarity between domain thumbnails \cite{alsum2014thumbnail}. This has previously been considered for pairs of images, although plotting similarity over time has yet to be considered. Several methods for comparing thumbnails have been proposed and implemented previously \cite{henzinger,broder,manku}, and an accessible summary is provided in \cite{alsum2014thumbnail}. Some image comparison techniques may not always produce meaningful results -- e.g., images on a homepage may change frequently, with no change in website content.

Therefore, we propose the use of the structural similarity index (SSIM) \cite{ssim}, which, broadly speaking, measures the similarity between two images by comparing average pixel intensity in various sub-windows of the page. The SSIM value is generally between $-1$ and $1$, with $1$ only achieved when two images are identical. In order to compare this metric directly with the differences calculated between text and website links, we scale the SSIM value to lie between $0$ and $1$, where $0$ indicates no change between a pair of images. We call this metric, $d$, a measure of difference between image thumbnails.

\begin{figure}[h!]
\centering
\def\svgwidth{
0.7\columnwidth}
\import{report-images/}{all-waybacks.pdf_tex}
  \caption{Three image comparisons from \texttt{www.ndp.ca}.}
\label{fig::wayback_images}
\end{figure}

We wrote code in \texttt{python} to automatically generate thumbnails from a chosen web domain at all points recorded in the Wayback Machine, and used this to obtain a library of 108 thumbnails from \texttt{www.ndp.ca} from 2005-2019. There were times when the Wayback Archive had only saved a page that had failed to render, which presented itself as a primarily white webpage. We detected these `fails' and removed them from the data set, by imposing a maximum percentage of white pixels (80\%). In Figure \ref{fig::wayback_images}, we display an example of a failed render, as well as two different timesteps which demonstrate visually the value of the SSIM metric.

\section{Links}

\section{Conclusion}

\bibliography{bibliography}
\bibliographystyle{ieeetr}


\end{document}
