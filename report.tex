\documentclass[10pt, a4paper]{article}
\usepackage{amsmath}
\usepackage{amsfonts}
\usepackage[english]{babel}
\usepackage[utf8]{inputenc}
\usepackage{color}
\usepackage{graphicx}
\usepackage[version=4]{mhchem}
\usepackage{rotating}
\renewcommand{\baselinestretch}{1}
\usepackage{fullpage}
\usepackage{mathtools}
\usepackage{caption}
\usepackage{tikz}
\usepackage{standalone}
\usetikzlibrary{decorations.pathreplacing}
\usetikzlibrary{arrows}
\usetikzlibrary{fadings}
\usepackage{parskip}
\setlength{\parindent}{0pt}

\begin{document}

\title{IPSW - Modelling Change of Website Archives}
\author{Group 4}
\date{\vspace{-5ex}}
\maketitle

\section{Introduction}

\subsection{Goal}
Our goal is to investigate a novel metric for domain change that focuses on three branches of indicators, namely, changes in webpage text, changes in the links extending from the webpage and change to the homepage thumbnails. We propose that the  magnitude  of the total change in the domain from time-step $n-1$ to time-step $n$ can be modelled as 
\begin{equation}
\sigma(n) = \left(1-\frac{\Delta\, \,\text{links}}{\Sigma\,\,\text{links}}\right)w_1 + (\text{change in text})w_2 + (\text{change in content management server})w_3,
\end{equation}
where $w_1$, $w_2$, and $w_3$ weight the relevant contributions of URL changes, text changes, and CMS changes, respectively.

\section{Obtaining the data}
\subsection{Political party data}
\subsection{Pan Am Games 2015}

\section{Text}
We obtain the text from the homepage at every point in time. Our goal is to compare the text from one time-point to the next and quantify how much the text has changed between these measurements. There are a variety of metrics within the literature, in particular we explore the metrics described by Kwon et~al.  \cite{kwon2006precise}.
\subsection{Metrics}
\subsubsection*{Byte-wise comparison metric}
Compares two webpages sequentially character by character. The metric returns 0 when no change and 1 otherwise. It returns 1 for even trivial cases, for example, blank space -- over-sensitive.

\subsubsection*{TF.IDF cosine distance}
Comparing the number of times key words appear in the document.

\subsubsection*{Word distance}
Percentage of words that have stayed the same.

\subsubsection*{Edit distance}
Number of edits that are required to transform one sentence into another.

\subsubsection*{Shingling metric}
Breaks up the webpage into subsequences called ``shingles'' that contain $k$ words.

\subsection{Issues}
\begin{itemize}
	\item Byte-wise is over-sensitive,
	\item TF.IDF and Word Distance cannot take into account change in word order,
	\item Shingling is over-sensitive to small webpages,
	
\end{itemize}


\section{Thumbnails}
A promising method to check whether meaningful changes have occurred to a domain is to compare homepage thumbnails at two different time points. This is often done manually, although automated image analysis approaches have also been investigated. This method may not always produce meaningful results -- e.g., moving a single image from a homepage may modify the appearance dramatically without a change in content. Therefore, it is important to also investigate alternative metrics. A summary of different approaches is provided in \cite{kwon2006precise}. {\color{red} Ian could you add in a lit review?}

\section{Links}

\section{Conclusion}

\bibliography{bibliography}
\bibliographystyle{ieeetr}


\end{document}